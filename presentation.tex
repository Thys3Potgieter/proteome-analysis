%\documentclass[mathserif]{beamer}
\documentclass{beamer}
\usepackage{graphicx}
\usepackage{hyperref}
\setbeamertemplate{footline}[frame number]

\title{Increased concentration of proteins with growth rate can result from passive resource redistribution}
\author{Uri Barenholz}

\begin{document}

\maketitle
\begin{frame}
\frametitle{Cell physiology changes dramatically across different growth conditions}
\begin{itemize}[<+->]
\item For glucose-limited chemostat growth\footnote{Data from Valgepea et. al. 2013} at rates ranging from $0.11 [h^{-1}]$ to $0.49 [h^{-1}]$
\begin{itemize}
\item Cell volume varies between $0.73 [fL]$ and $1.69 [fL]$.
\item Protein fraction out of the cell dry weight changes from $61\%$ to $50\%$.
\item Proteome composition changes by an average CV of $40\%$.
\end{itemize}
\item Similar changes are observed under varying carbon sources
\footnote{Heinemann et. al. unpublished data}
\end{itemize}
\end{frame}

\begin{frame}
\frametitle{Can growth rate characterize and predict the observed changes?}
\begin{itemize}[<+->]

\item Cell size increases with growth rate.
\item Protein fraction out of cell dry weight decreases with growth rate due to increased ribosomes concentration.
\item But what about proteome composition?
\end{itemize}
\end{frame}

\begin{frame}
\frametitle{Pearson correlation of a protein concentration with growth rate reveals their inter-dependence}
\begin{figure}[h!]
\centering
\includegraphics[scale=0.9]{SingleProtK00549.pdf}
\end{figure}
\end{frame}

\begin{frame}
\frametitle{Increasing concentration with growth rate is a dominant trend in proteomic data}
\begin{figure}[h!]
\centering
\includegraphics[scale=0.9]{GrowthRateCorrelationVal.pdf}
\end{figure}
\end{frame}

\begin{frame}
\frametitle{Increasing concentration with growth rate is a dominant trend in proteomic data}
\begin{figure}[h!]
\centering
\includegraphics[scale=0.9]{GrowthRateCorrelation.pdf}
%% should split to two graphs - one for each data set.
\end{figure}
\end{frame}

\begin{frame}
\frametitle{Proteins sharing positive correlation are not necessarily coordinated}
\begin{figure}[h!]
\centering
\href{https://plot.ly/~uri.barenholz/0/protein-concentration-vs-growth-rate/}
    {\includegraphics[scale=0.9]{Noncorrelated.pdf}}
\end{figure}
\end{frame}

\begin{frame}
\frametitle{Normalization can be applied to identify coordination between proteins}
%Assuming $g_i$ are the growth rates, $p_i$ are the concentrations of one protein at these growth rates, and $q_i$ are the concentrations of another protein at these growth rates, the proteins are coordinated if for all $i$, $\frac{p_i}{q_i}=C$ some constant that is independent of $i$.
%This also implies that $\bar{p_i}=\bar{q_i}C$.
%Therefore, normalizing each protein by assigning $p'_i=p_i/\bar{p_i}$ and $q'_i=q_i/\bar{q_i}$ results in $p'_i=q'_i$
Specifically, for linear correlations, the linear parameters should be identical.
\end{frame}

\begin{frame}
\frametitle{Proteins concentrations increase proportionately with growth}
\begin{figure}[h!]
\centering
\includegraphics[scale=0.9]{AllProtsNormalizedSlopes.pdf}
\end{figure}
\end{frame}

\begin{frame}
\frametitle{Ribosomal proteins are coordinated with a large number of other proteins that increase with growth}
\begin{figure}[h!]
\centering
\includegraphics[scale=0.9]{AllProtsVSRibosomalNoExpNormalizedSlopes.pdf}
\end{figure}
\end{frame}

\begin{frame}
\frametitle{The observed distribution deviates from the expected one, given estimated experimental noise}
\begin{figure}[h!]
\centering
\includegraphics[scale=0.9]{AllProtsVSRibosomalNormalizedSlopes.pdf}
\end{figure}
\end{frame}

\begin{frame}
\frametitle{An analogy to a country's budget can give insights to observed phenomena}
%\begin{figure}[h!]
%\centering
%\includegraphics[scale=0.75]{SimpleBudget.pdf}
%\end{figure}
\end{frame}

%\begin{frame}
%\frametitle{An analogy to a country's budget can give insights to observed phenomena}
%\begin{figure}[h!]
%\centering
%\includegraphics[scale=0.75]{SimpleBudget2.pdf}
%\end{figure}
%\end{frame}

\begin{frame}
\frametitle{Linear, coordinated increase in concentration with growth rate is an expected result of passive resource redistribution}
\begin{itemize}[<+->]
\item Improved growth conditions eliminate the need for expression of specific proteins.
\item Freed translational and transcriptional resources are evenly distributed across the expression program of the cell.
\item The growth rate increases as the ratio of macro-molecular synthesizing components to the biomass increases.
\item the predicted concentration of essential components at zero growth rate is 0.
\end{itemize}
\end{frame}
\begin{frame}
\frametitle{The concentration of proteins that are highly positively correlated with growth rate does not drop to zero at zero growth}
\begin{figure}[h!]
\centering
\includegraphics[scale=0.9]{GlobalClusterGRFit.pdf}
\end{figure}
\end{frame}
\begin{frame}
\frametitle{Non-zero concentration at zero growth can result from protein degradation}
\begin{itemize}[<+->]
\item Even at zero growth rate, macromolecules are synthesized and degraded.
\item The macromolecular synthesis rate is the growth rate plus the degradation rate.
\item The proteome composition reflects the macromolecular synthesis rate and not the observed growth rate.
\item The experimental data suggests an average half life time of $\approx 1.7[h]$ for macro molecules.
\end{itemize}
\end{frame}

\begin{frame}
\frametitle{Linear, coordinated correlation accounts only for a small fraction of the observed variability in proteome composition}
\begin{figure}[h!]
\centering
\includegraphics[scale=0.9]{ExpVar3.pdf}
\end{figure}
\end{frame}

\begin{frame}
\frametitle{Linear, coordinated correlation accounts only for a small fraction of the observed variability in proteome composition}
\begin{itemize}[<+->]
\item Proteins with high abundance have high contribution to proteome variability but are also likely to have specific regulation.
\item Assuming some proteins increase in concentration requires others to decrease, but knowing which is non-trivial.
\item Unknown level of experimental noise may account for much of the variability.
\end{itemize}
\end{frame}

\begin{frame}
\frametitle{Conclusions}
\begin{itemize}
\item Many proteins are positively highly correlated with growth rate, even under different carbon sources.
\begin{itemize}
\item 296 out of 919 in chemostat conditions, 881 out of 1654 under different carbon sources.
\end{itemize}
\item The normalized slopes of the responses of these proteins are relatively similar.
\item However, for current data sets, correlation with growth rate does not explain a significant fraction of overall variability observed in the proteome.
\begin{itemize}
\item Only $5-10\%$ of total variability is explained by linear correlation with growth rate.
\end{itemize}
\item Lacking information regarding experimental noise implies one cannot distinguish between model validity and measurement errors.
\end{itemize}
\end{frame}

\begin{frame}
\frametitle{Future directions}
\end{frame}

\begin{frame}
\frametitle{Collecting proteomics data to further investigate existing model and other phenomena}
\begin{itemize}
\item Existing proteomics data sets lack error estimations and span relatively narrow growth rates.
\item Our lab is currently constructing a multi chemostat/turbidostat system that can be used to collect samples in exponential growth under various conditions.
\item The INCPM has whole proteome measuring capabilities.
\item Technical constraints limit the ability to span many conditions at low noise levels.
\item No ability to compare amounts of different peptides across the same sample.
\end{itemize}
\end{frame}

\begin{frame}
\frametitle{Investigate mechanisms underlying the Monod equation using proteomics data}
\begin{itemize}
\item A few mechanisms have been suggested to underly the Monod relation of growth rate to nutrient concentration.
\item Different mechanisms are expected to have different proteomic footprints.
\item Our in-construction turbidostat/chemostat system can be used to collect samples of cells on different points along the Monod curve.
\item Modern enzymatic essays can be used to quantify the low nutrient concentrations resulting in slow growth rates.
\item Collecting proteomic data on growth cultures may help pin-point which models are dominant for different nutrients.
\end{itemize}
\end{frame}
\begin{frame}
\frametitle{Quantitatively model central dogma and major cellular components in-silico for sensitivity analysis and exploration of the phenotypic space}
\begin{itemize}
\item Current cellular models are either simple, but constrained to limited aspects or cellular subsystems, or extremely complicated and verbose
\item A gross, whole cell model can include major cellular components and processes and link them via simple equations
\item Such a model can be used to quantify the sensitivity to different biological parameters and to explore the phenotypic space
\end{itemize}
\end{frame}
\begin{frame}
\frametitle{Compare expected effects of different ribosomal characteristics on ribosome concentration as measured in existing data sets}
\begin{itemize}
\item Different physiological aspects such as variability in translation rates, degradation of ribosomes and proteins, and minimal concentration of ribosomes in cells may affect optimal ribosome concentration
\item Existing experimental measurements of ribosomes concentration under different growth conditions can be compared with optimal assumptions, as predicted by the interplay of such factors
\item These comparisons may help quantify the extent to which such different aspects play an actual role in determining the ribosomal concentration in vivo 
\end{itemize}
\end{frame}

\begin{frame}
\frametitle{Explore population dynamics in the gut microbiome by inferring growth rates from peak to trough ratios in sequencing data}
\begin{itemize}
\item Recent results from the Segal lab show varying peak to trough ratios in sequencing data
\item The peak to trough ratio is suggested to reflect the number of replication origins and can thus be used to estimate the growth rate
\item Initial results suggest a much faster growth rate in samples than the expected $\approx 1-2$ day doubling time in the gut
\item Modelling the different growth rates and relative abundances of sub populations may thus increase our understanding of population dynamics in the gut
\end{itemize}
\end{frame}
\begin{frame}
\frametitle{Characterize possible metabolic network motifs arising from the concavity of reaction kinetics}
\begin{itemize}
\item Reaction rates tend to monotonically and concavely increase as a function of substrate concentration
\item This characteristic constrains the available responses to changes in metabolite levels around steady state levels
\item Combining these constraints with biological objective functions may help characterize kinetic parameters around steady state points and possibly also metabolic network motifs
\end{itemize}
\end{frame}

\begin{frame}
\frametitle{Derive theoretical bounds on chemical reaction rates}
\begin{itemize}
\item While catalysis is a cornerstone in every  biological process, very little is known about the theoretical bounds on catalysis rate and efficiency under given physiological constrains
\item Obtaining such constraints for various reactions may shed light on the relative efficiency of biological systems compared with theoretical bounds as well as the constraints biological systems face
\end{itemize}
\end{frame}
\end{document}
