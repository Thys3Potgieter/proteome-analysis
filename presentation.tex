%\documentclass[mathserif]{beamer}
\documentclass{beamer}
\usepackage{graphicx}

\title{Increased concentration of proteins with growth rate can result from passive resource redistribution}
\author{Uri Barenholz}

\begin{document}

\maketitle
\begin{frame}
\frametitle{Cell physiology changes dramatically across different growth conditions}
For glucose-limited chemostat growth at rates ranging from $0.11 [h^{-1}]$ to $0.49 [h^{-1}]$
\begin{itemize}
\item Cell volume varies between $0.73 [fL]$ and $1.69 [fL]$.
\item Protein fraction out of the cell dry weight changes from $61\%$ to $50\%$.
\item Proteome composition changes by an average CV of $40\%$.
\end{itemize}
\end{frame}

\begin{frame}
\frametitle{Can growth rate characterize and predict the observed changes?}
\begin{itemize}
\item Cell size increases with growth rate.
\item Protein fraction out of cell dry weight decreases with growth rate due to increased ribosomes concentration.
\item And what about proteome composition?
\end{itemize}
\end{frame}

\begin{frame}
\frametitle{Increasing concentration with growth rate is a dominant trend in proteomic data}
\begin{figure}[h!]
\centering
\includegraphics{GrowthRateCorrelation.pdf}
%% should split to two graphs - one for each data set.
\end{figure}
\end{frame}

\begin{frame}
\frametitle{Shared positive correlation does not imply coordinated response}
\begin{figure}[h!]
\centering
\includegraphics{Noncorrelated.pdf}
\end{figure}
\end{frame}

\begin{frame}
\frametitle{Proteins increase proportionately with growth}
\end{frame}

\begin{frame}
\frametitle{Ribosomal proteins are coordinated with a large number of other proteins that increase with growth}
\end{frame}

\begin{frame}
\frametitle{The observed distribution deviates from the expected one, given estimated experimental noise}
\end{frame}

\begin{frame}
\frametitle{An analogy to a country's budget can give insights to observed phenomena}
\end{frame}

\begin{frame}
\frametitle{Linear, coordinated increase in concentration with growth rate is an expected result of passive resource redistribution}
\begin{itemize}
\item Improved growth conditions eliminates the need for expression of specific proteins.
\item Freed translational and transcriptional resources are evenly distributed across the expression program of the cell.
\item The growth rate increases as the ratio of macro-molecular synthesizing components to the biomass increases.
\end{itemize}
\end{frame}

\begin{frame}
\frametitle{Non-zero intercept can result from protein degradation}
\end{frame}

\begin{frame}
\frametitle{However, linear, coordinated correlation accounts only for a small fraction of the observed data}
%% possible reasons are experimental noise and false assumptions
\end{frame}
\begin{frame}
\frametitle{Conclusions}
\end{frame}

\end{document}
