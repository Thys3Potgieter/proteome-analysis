%\documentclass[mathserif]{beamer}
\documentclass{beamer}
\usepackage{graphicx}
\usepackage{hyperref}
\setbeamertemplate{footline}[frame number]

\title{Increased concentration of proteins with growth rate can result from passive resource redistribution}
\author{Uri Barenholz}

\begin{document}

\maketitle
\begin{frame}
\frametitle{Cell physiology changes dramatically across different growth conditions}
\begin{itemize}[<+->]
\item For glucose-limited chemostat growth\footnote{Data from Valgepea et. al. 2013} at rates ranging from $0.11 [h^{-1}]$ to $0.49 [h^{-1}]$
\begin{itemize}
\item Cell volume varies between $0.73 [fL]$ and $1.69 [fL]$.
\item Protein fraction out of the cell dry weight changes from $61\%$ to $50\%$.
\item Proteome composition changes by an average CV of $40\%$.
\end{itemize}
\item Similar changes are observed under varying carbon sources
\footnote{Heinemann et. al. unpublished data}
\end{itemize}
\end{frame}

\begin{frame}
\frametitle{Can growth rate characterize and predict the observed changes?}
\begin{itemize}[<+->]

\item Cell size increases with growth rate.
\item Protein fraction out of cell dry weight decreases with growth rate due to increased ribosomes concentration.
\item But what about proteome composition?
\end{itemize}
\end{frame}

\begin{frame}
\frametitle{Pearson correlation of a protein concentration with growth rate reveals their inter-dependence}
\begin{figure}[h!]
\centering
\includegraphics[scale=0.9]{SingleProtK00549.pdf}
\end{figure}
\end{frame}

\begin{frame}
\frametitle{Increasing concentration with growth rate is a dominant trend in proteomic data}
\begin{figure}[h!]
\centering
\includegraphics[scale=0.9]{GrowthRateCorrelationVal.pdf}
\end{figure}
\end{frame}

\begin{frame}
\frametitle{Increasing concentration with growth rate is a dominant trend in proteomic data}
\begin{figure}[h!]
\centering
\includegraphics[scale=0.9]{GrowthRateCorrelation.pdf}
%% should split to two graphs - one for each data set.
\end{figure}
\end{frame}

\begin{frame}
\frametitle{Proteins sharing positive correlation are not necessarily coordinated}
\begin{figure}[h!]
\centering
\href{https://plot.ly/~uri.barenholz/0/protein-concentration-vs-growth-rate/}
    {\includegraphics[scale=0.9]{Noncorrelated.pdf}}
\end{figure}
\end{frame}

\begin{frame}
\frametitle{Normalization can be applied to identify coordination between proteins}
%Assuming $g_i$ are the growth rates, $p_i$ are the concentrations of one protein at these growth rates, and $q_i$ are the concentrations of another protein at these growth rates, the proteins are coordinated if for all $i$, $\frac{p_i}{q_i}=C$ some constant that is independent of $i$.
%This also implies that $\bar{p_i}=\bar{q_i}C$.
%Therefore, normalizing each protein by assigning $p'_i=p_i/\bar{p_i}$ and $q'_i=q_i/\bar{q_i}$ results in $p'_i=q'_i$
Specifically, for linear correlations, the linear parameters should be identical.
\end{frame}

\begin{frame}
\frametitle{Proteins concentrations increase proportionately with growth}
\begin{figure}[h!]
\centering
\includegraphics[scale=0.9]{AllProtsNormalizedSlopes.pdf}
\end{figure}
\end{frame}

\begin{frame}
\frametitle{Ribosomal proteins are coordinated with a large number of other proteins that increase with growth}
\begin{figure}[h!]
\centering
\includegraphics[scale=0.9]{AllProtsVSRibosomalNoExpNormalizedSlopes.pdf}
\end{figure}
\end{frame}

\begin{frame}
\frametitle{The observed distribution deviates from the expected one, given estimated experimental noise}
\begin{figure}[h!]
\centering
\includegraphics[scale=0.9]{AllProtsVSRibosomalNormalizedSlopes.pdf}
\end{figure}
\end{frame}

\begin{frame}
\frametitle{An analogy to a country's budget can give insights to observed phenomena}
%\begin{figure}[h!]
%\centering
%\includegraphics[scale=0.75]{SimpleBudget.pdf}
%\end{figure}
\end{frame}

%\begin{frame}
%\frametitle{An analogy to a country's budget can give insights to observed phenomena}
%\begin{figure}[h!]
%\centering
%\includegraphics[scale=0.75]{SimpleBudget2.pdf}
%\end{figure}
%\end{frame}

\begin{frame}
\frametitle{Linear, coordinated increase in concentration with growth rate is an expected result of passive resource redistribution}
\begin{itemize}[<+->]
\item Improved growth conditions eliminate the need for expression of specific proteins.
\item Freed translational and transcriptional resources are evenly distributed across the expression program of the cell.
\item The growth rate increases as the ratio of macro-molecular synthesizing components to the biomass increases.
\item the predicted concentration of essential components at zero growth rate is 0.
\end{itemize}
\end{frame}
\begin{frame}
\frametitle{The concentration of proteins that are highly positively correlated with growth rate does not drop to zero at zero growth}
\begin{figure}[h!]
\centering
\includegraphics[scale=0.9]{GlobalClusterGRFit.pdf}
\end{figure}
\end{frame}
\begin{frame}
\frametitle{Non-zero concentration at zero growth can result from protein degradation}
\begin{itemize}[<+->]
\item Even at zero growth rate, macromolecules are synthesized and degraded.
\item The macromolecular synthesis rate is the growth rate plus the degradation rate.
\item The proteome composition reflects the macromolecular synthesis rate and not the observed growth rate.
\item The experimental data suggests an average half life time of $\approx 1.7[h]$ for macro molecules.
\end{itemize}
\end{frame}

\begin{frame}
\frametitle{Linear, coordinated correlation accounts only for a small fraction of the observed variability in proteome composition}
\begin{figure}[h!]
\centering
\includegraphics[scale=0.9]{ExpVar3.pdf}
\end{figure}
\end{frame}

\begin{frame}
\frametitle{Linear, coordinated correlation accounts only for a small fraction of the observed variability in proteome composition}
\begin{itemize}[<+->]
\item Proteins with high abundance have high contribution to proteome variability but are also likely to have specific regulation.
\item Assuming some proteins increase in concentration requires others to decrease, but knowing which is non-trivial.
\item Unknown level of experimental noise may account for much of the variability.
\end{itemize}
\end{frame}

\begin{frame}
\frametitle{Conclusions}
\begin{itemize}
\item Many proteins are positively highly correlated with growth rate, even under different carbon sources.
\begin{itemize}
\item 296 out of 919 in chemostat conditions, 881 out of 1654 under different carbon sources.
\end{itemize}
\item The normalized slopes of the responses of these proteins are relatively similar.
\item However, for current data sets, correlation with growth rate does not explain a significant fraction of overall variability observed in the proteome.
\begin{itemize}
\item Only $5-10\%$ of total variability is explained by linear correlation with growth rate.
\end{itemize}
\item Lacking information regarding experimental noise implies one cannot distinguish between model validity and measurement errors.
\end{itemize}
\end{frame}
\begin{frame}
\frametitle{Questions and future directions}
\end{frame}
\end{document}
