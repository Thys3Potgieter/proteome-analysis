\documentclass[a4page,notitlepage]{article}
\usepackage{color,soul,amsmath,graphicx}
\usepackage[outercaption]{sidecap}
\usepackage[citestyle=authoryear]{biblatex}
\usepackage[font=small,labelfont=bf]{caption}
\usepackage{verbatim}
\sidecaptionvpos{figure}{c}
\providecommand{\abs}[1]{\lvert#1\rvert}
\providecommand{\norm}[1]{\lVert#1\rVert}

\title{Proteome Analysis Results Section}
\author{Uri Barenholz}
\date{April 2014}

\begin{document}
\maketitle
\section{Results}

\begin{figure}[h]
\centering
\includegraphics{GrowthRateCorrelation.pdf}
\caption{
A significant fraction of the proteins have positive correlation with the growth rate.
These proteins span all the functional groups.
Proteins with unknown function show less correlation with growth rate (as well as proteins with low levels of expression, data not shown).
}
\label{growth-corr}
\end{figure}

\begin{figure}[h]
\centering
\includegraphics{GlobalClusterGRFit.pdf}
\caption{
The global cluster (sum of all proteins with a correlation in the rage 0.4 to 0.8 with the growth rate) is reasonably correlated with growth rate.
Both weighted sum and normalized sum are presented.
Weighted sum means the concentrations of all proteins in the group are summed.
Normalized sum means every protein is first normalized to have an average concentration of 1 across the different growth conditions, and then all proteins in the group are summed.
Some of the unexplained variability of the global cluster by the growth rate can indicate errors in growth rate measurements and/or differences in degredation rates across conditions.
}
\label{global-grcorr}
\end{figure}

\begin{figure}[h]
\centering
\includegraphics{AllProtsVSRibosomalNormalizedSlopes.pdf}
\caption{
    (A) Most proteins have similar, positive, response to the growth rate meaning they maintain their relative proportions across conditions.
    (B) The response of all of the proteins peaks at the same values as the response of the ribosomal proteins.
}
\label{global-fit}
\end{figure}

\begin{comment}
\begin{figure}[h]
\centering
\includegraphics{CoordinatedRSquareComparison.pdf}
\caption{
  Proteins in the global cluster fit reasonably well to the global cluster itself
}
\label{global-fit}
\end{figure}

\begin{figure}[h]
\centering
\includegraphics{GlobalClusterCorr.pdf}
\caption{
Proteins that have a high corrleation (0.4-0.8) with growth rate mostly have even higher correlation to the sum of these proteins (both weighted sum and normalized sum are presented).
Weighted sum means the concentrations of all proteins in the group are summed.
Normalized sum means every protein is first normalized to have an average concentration of 1 across the different growth conditions, and then all proteins in the group are summed.
The higher correlation indicates that their response is coordinated (they scale by the same factor between conditions).
}
\label{global-corr}
\end{figure}

\begin{figure}[h]
\centering
\includegraphics{GlobalClusterRSquare.pdf}
\caption{
Plotting the $r^2$ distribution shows that a large fraction of the variability of these proteins is captured by the global response.
}
\label{global-rsq}
\end{figure}

\begin{itemize}
\item This happens in multiple organisms and datasets (show yeast, two datasets of coli). Possibly add mRNA measurements ??.
\end{itemize}
\end{comment}
\end{document}

